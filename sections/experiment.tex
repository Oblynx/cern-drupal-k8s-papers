\subsection{Service level indicators}

- entire infra: 250 req/sec
- critical website: 35 req/sec
- standard website: 5-10 req/sec % TODO set final value
- test website: 2 max threads

- Find configuration that is able to handle specified req/sec load
- Two Graphs (1 for critical and 1 normal), showing AVG response time plots + Request per second plots
- See the Resources usage for each configuration
- Make an estimation of total resources , note that critical websites will have $(MAX_LOAD) + 2(IDLE_LOAD)$ since they will have 3 pods

Total number of Critical websites: 20
Total number of Standard websites: 600
Total number of Test websites: 500


- Make an estimation of total infra resources ( sum to the estimation of total resources with openshift-* elements)

% #### Resources used in the physical infra:
% - Memory: 2TB
% - CPU: 512

\subsection{Stress test setup}


 (we have provioned websites and migrated data from the old infra)
\subsubsection*{Stress load}

To evaluate the behavior of our Kubernetes infrastructure, with the goal of analysing the new infrastructure's capability of answering the same needs with fewer resources, the following experiment was conducted.

The following elements were required, 

The experiment has a dedicated Kubernetes cluster that deploys a custom tool that will simulate users. 
The simulation of users is done by running multiple \hyperlink{https://kubernetes.io/docs/concepts/workloads/pods/}{Pods}, each containing multiple processes that will request URLs at random\footnote{It iterates urls through the website to discover all the URLs, and only after having all the URLs, it connects to one of them at random}.

Multiple runs have been made with different number of parallel users per .

We observed that after a number of parallel users in the same pod, the users would get bottle-necked by Pod network constraints. So for our tests we observer that each Pod could have no more than 5 users before network constraints would interfere.

Important notes:
- RAM consumption before first connection is ~11MiB, 500 to 600MiB after (even if no accesses are being done)
- Time used for a stress run was decided to be 6min because it's enough to see the stabilized response time

\subsubsection*{Measurements}

\subsection{Results: resource provision}
