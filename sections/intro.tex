Google rewrites most of their software every few years \cite{henderson_software_2020}.
This surprising activity consumes a large fraction of their resources each year!
And yet, they consider it crucial to their agility and long-term success, because software requirements change as technologies evolve -- and with them, user expectations.
This practice typically reduces unnecessary complexity in each iteration,
and transfers knowledge into the new generation of engineers.

All these factors apply as much to CERN as they apply to Google.
Despite the much slower pace at which services evolve, CERN lives in the same dynamic technological environment.
Without a constant input of effort our software falls behind and fails to address modern expectations
in such aspects as security, high availability, portability and isolation.
At the same time unnecessary complexity accumulates:
yesterday's custom solutions can often be replaced with new upstream components,
the product of continuous standardization of solutions to problems that affect entire industries.

The increasing divergence between the original requirements the specified a piece of software and present requirements
results in \emph{technical debt} \cite{fairbanks_ur-technical_2020}.
The main purpose of this work is to pay back technical debt in CERN's Content Management Systems by modernizing the software architecture
and making the service more secure and flexible.


\subsection{Introduction to the concepts in this work}

\subsubsection*{Why Kubernetes?}

Kubernetes is for cloud native applications an extension of what the operating system is for traditional applications.
This is highlighted by the moniker "cloud operating system".
It is becoming a de facto standard for Platform as a Service, abstracting computational infrastructure and making applications deployable to sites across the globe.



- common platform across web frameworks
- common components, reuse, expertise sharing
- isolation

Mention Operators.

\subsubsection*{What is Drupal?}
\label{what-is-drupal}

Drupal is an open-source content management system (CMS): a tool for site builders to organize and deliver content to their website visitors.
It's used in 10\% of the top-10k websites with the highest traffic \cite{builtwith_pty_ltd_open_nodate},\cite{q-success_di_gelbmann_gmbh_wordpress_nodate}.

% TODO
Who is the intended audience that can benefit from a Drupal infrastructure?
- an organization that needs a dynamic Drupal environment, with lots of site turnover
- eg: universities, organizations comprising many departments and independent activities
% TODO

The market leader in Drupal's niche is the CMS WordPress \cite{builtwith_pty_ltd_open_nodate}.
Drupal is often contrasted with it and, according to UX surveys conducted by a Drupal-focused company, seems to offer a complicated start for beginner users,
but a powerful experience for experts \cite{buytaert_state_nodate}.

Drupal is frequently embraced for being an open source community driven project, which makes it strategically attractive for enterprise sustainability \cite{cern_geneva_real_2019}.
Use cases range from simple blogs to professional newspaper publishing, from enterprise presentation to e-commerce, across government and private sector entities \cite{drupal_community_explore_nodate}.
A frequently cited feature is the flexibility with which it adapts to bespoke requirements, while scaling to large amounts of content.

\subsubsection*{Drupal at CERN}
\label{drupal-at-cern}

Content Management Systems -> Drupal

The Drupal service at CERN was established in 2011, responding to a need for a standardized and user-friendly website environment in the Organization.
Before this, the websites ecosystem was technologically diverse, lacking standards and institutional support.
The tool was selected over competitors due its extensibility, existing contributed modules and active community.

At CERN, Drupal has become the platform of choice for public outreach.
We have developed in-house expertise on both Drupal development (with custom themes and components) and hosting.

\subsection{Article structure}

Having laid down background information on the motivation, technologies and concepts used in this work, in the following sections we will describe:

\begin{tabular}{l|l}
    Section \ref{sec-drupalsvc} & Content Management as a Service at CERN: \emph{what problem are solving?} \\
    Section \ref{sec-phys-infra} & The current implementation of the Drupal service \\
    Section \ref{sec-k8s-design} & The Kubernetes design of its successor, currently in Pilot \\
    Section \ref{sec-experiment} & A preliminary experimental validation of the new platform \\
    Section \ref{sec-discussion} & Plans for the future and the nuances of this work \\
\end{tabular}