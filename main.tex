%%%%%%%%%%%%%%%%%%%%%%% file template.tex %%%%%%%%%%%%%%%%%%%%%%%%%
%
% This is a template file for Web of Conferences Journal
%
% Copy it to a new file with a new name and use it as the basis
% for your article
%
%%%%%%%%%%%%%%%%%%%%%%%%%% EDP Science %%%%%%%%%%%%%%%%%%%%%%%%%%%%
%
%%%\documentclass[option]{webofc}
%%% "twocolumn" for typesetting an article in two columns format (default one column)
%
\documentclass{webofc}
\usepackage[varg]{txfonts}   % Web of Conferences font
%
% Put here some packages required or/and some personal commands
%
%
\usepackage{hyperref}
\hypersetup{
    colorlinks=true,
    linkcolor=blue,
    urlcolor=cyan,
}
\usepackage{wrapfig}
\usepackage{tabularx}
\usepackage{svg}
\usepackage{natbib}
\definecolor{amethyst}{rgb}{0.6, 0.4, 0.8}
\definecolor{asparagus}{rgb}{0.53, 0.66, 0.42}
\definecolor{darkseagreen}{rgb}{0.56, 0.74, 0.56}
\definecolor{carolinablue}{rgb}{0.6, 0.73, 0.89}
\definecolor{beige}{rgb}{0.80, 0.78, 0.63}
\definecolor{fluorescentorange}{rgb}{.90, 0.65, 0.0}

\begin{document}
%\title{CERN's 1000 Drupal Websites Move to Kubernetes}
\title{Building a Kubernetes infrastructure for CERN's Content Management Systems}
% NOTE: Drupal is probably not a relevant keyword for our audience to be in the title.
\author{\firstname{Konstantinos} \lastname{Samaras-Tsakiris}\inst{1}\fnsep\thanks{\email{konstantinos.samaras-tsakiris@cern.ch}} \and
        \firstname{Rajula} \lastname{Vineet Reddy}\inst{1}\fnsep \and
        \firstname{Francisco} \lastname{Borges Aurindo Barros}\inst{1,2}\fnsep \and
        \firstname{Eduardo} \lastname{Alvarez Fernandez}\inst{1}\fnsep \and
        \firstname{Andreas} \lastname{Wagner}\inst{1}
        % etc.
}

\institute{CERN \and Instituto Superior Técnico}

\abstract{%
The infrastructure behind \href{https://home.cern}{home.cern} and 1000 other Drupal websites serves more than 15,000 unique visitors daily.
To best serve the site owners, a small engineering team needs development speed to adapt to their evolving needs and operational velocity to troubleshoot emerging problems rapidly.
We designed a new Web Frameworks platform by extending Kubernetes to replace the ageing physical infrastructure and reduce the dependency on homebrew components.

The new platform is modular, built around standard components and thus less complex to operate.
Some requirements are covered solely by upstream open source projects, 
whereas others by components shared across CERN's web hosting platforms.
We leverage the \href{https://sdk.operatorframework.io/docs/}{Operator framework} and the Kubernetes API
to get observability, policy enforcement, access control and auditing, and high availability for free.
Thanks to containers and namespaces, websites are isolated.
This isolation clarifies security boundaries and minimizes attack surface, while empowering site owners.

% Explain better that the "operational insights" aren't here yet, but will be at the conference
In this work we present the open-source design of the new system and contrast it with the one it replaces, demonstrating how we drastically reduced our technical debt.
As the project enters its pilot phase, at CHEP 2021 we will also share operational insights.
}
%
\maketitle
%
\section{Introduction}
\label{intro}
Intro.
Problem statement. Change infra -> kubernetes, reduce tech debt.
Google rewrites software every few years.

Motivations, outline.

/subsection{Why Kubernetes?}

- common platform across web frameworks
- common components, reuse, expertise sharing
- isolation

Mention Operators.

%\section{What is required for Content Management as a Service?}
\section{Requirements for Content Management as a Service}
\label{sec-drupalsvc}
The Drupal service at CERN supports website admins to host and administer websites directed to the grand public,
such as experiment or departmental central websites.
Some of the most popular sites based on this service are \href{https://home.cern/}{home.cern}, \href{https://atlas.cern}{atlas.cern},
\href{https://cms.cern}{cms.cern}, \href{https://careers.cern}{careers.cern} and \href{https://visit.cern}{visit.cern}.
As such they form CERN's main outreach channel and are critical for the Organization's reputation.

The responsibility of site building at CERN often falls on administrative personnel, or personnel with little technical background in web technologies.
This in turn shapes the kind of service we provide; contrary to best practices, it is for example impractical to rely exclusively on developer-centric workflows like GitOps and CLI tools.
A small fraction of our user base, however, have indeed web development experience.

The consequence is that the Drupal service has a dual mission:
\begin{enumerate}
\item to ensure the high availability of these communication channels
\item to simplify administration and site building for the wide-ranging user base
\end{enumerate}

This work describes an infrastructure project that focuses on furthering the first mission, without sacrificing the second.
Ideally, the changes should be almost transparent for non-technical site administrators, while enabling previously unavailable best-practices workflows for technical users.


\subsection{Load characteristics}

\begin{figure}[h]
\centering
\includegraphics[width=1cm,clip]{/figures/}
\caption{Please write your figure caption here}
\label{fig-1}       % Give a unique label
\end{figure}
<graph: kibana top10 uniqueIP stacked bar> %  https://monit-timber-drupal-prod8.cern.ch/kibana/app/kibana#/visualize/create?type=histogram&indexPattern=d2ef41c0-e515-11e9-886e-69c303354aa9&_g=(refreshInterval:(pause:!t,value:0),time:(from:now-1M,to:now))&_a=(filters:!(),linked:!f,query:(language:kuery,query:'data.program:%22httpd%22'),uiState:(),vis:(aggs:!((enabled:!t,id:'1',params:(field:data.clientip),schema:metric,type:cardinality),(enabled:!t,id:'2',params:(field:data.sitename,missingBucket:!f,missingBucketLabel:Missing,order:desc,orderBy:'1',otherBucket:!t,otherBucketLabel:Other,size:10),schema:group,type:terms)),params:(addLegend:!t,addTimeMarker:!f,addTooltip:!t,categoryAxes:!((id:CategoryAxis-1,labels:(show:!t,truncate:100),position:bottom,scale:(type:linear),show:!t,style:(),title:(),type:category)),grid:(categoryLines:!f),legendPosition:right,seriesParams:!((data:(id:'1',label:'Unique%20count%20of%20data.clientip'),drawLinesBetweenPoints:!t,mode:stacked,show:true,showCircles:!t,type:histogram,valueAxis:ValueAxis-1)),times:!(),type:histogram,valueAxes:!((id:ValueAxis-1,labels:(filter:!f,rotate:0,show:!t,truncate:100),name:LeftAxis-1,position:left,scale:(mode:percentage,type:linear),show:!t,style:(),title:(text:'Unique%20count%20of%20data.clientip'),type:value))),title:'New%20Visualization',type:histogram))

<graph: kibana top10 requests stacked bar> %  https://monit-timber-drupal-prod8.cern.ch/kibana/app/kibana#/visualize/create?type=histogram&indexPattern=d2ef41c0-e515-11e9-886e-69c303354aa9&_g=(filters:!(),refreshInterval:(pause:!t,value:0),time:(from:now-1M,to:now))&_a=(filters:!(('$state':(store:appState),meta:(alias:!n,disabled:!f,index:d2ef41c0-e515-11e9-886e-69c303354aa9,key:data.sitename,negate:!t,params:(query:test-jardin-de-particules-d8.web.cern.ch),type:phrase,value:test-jardin-de-particules-d8.web.cern.ch),query:(match:(data.sitename:(query:test-jardin-de-particules-d8.web.cern.ch,type:phrase)))),('$state':(store:appState),meta:(alias:!n,disabled:!f,index:d2ef41c0-e515-11e9-886e-69c303354aa9,key:data.sitename,negate:!t,params:(query:test-uo-bat.web.cern.ch),type:phrase,value:test-uo-bat.web.cern.ch),query:(match:(data.sitename:(query:test-uo-bat.web.cern.ch,type:phrase)))),('$state':(store:appState),meta:(alias:!n,disabled:!f,index:d2ef41c0-e515-11e9-886e-69c303354aa9,key:data.sitename,negate:!t,params:(query:test-lpcc.web.cern.ch),type:phrase,value:test-lpcc.web.cern.ch),query:(match:(data.sitename:(query:test-lpcc.web.cern.ch,type:phrase)))),('$state':(store:appState),meta:(alias:!n,disabled:!f,index:d2ef41c0-e515-11e9-886e-69c303354aa9,key:data.sitename,negate:!t,params:(query:test-missionlhc.web.cern.ch),type:phrase,value:test-missionlhc.web.cern.ch),query:(match:(data.sitename:(query:test-missionlhc.web.cern.ch,type:phrase))))),linked:!f,query:(language:kuery,query:'data.program:%22httpd%22'),uiState:(),vis:(aggs:!((enabled:!t,id:'1',params:(),schema:metric,type:count),(enabled:!t,id:'2',params:(field:data.sitename,missingBucket:!f,missingBucketLabel:Missing,order:desc,orderBy:'1',otherBucket:!t,otherBucketLabel:Other,size:10),schema:group,type:terms)),params:(addLegend:!t,addTimeMarker:!f,addTooltip:!t,categoryAxes:!((id:CategoryAxis-1,labels:(show:!t,truncate:100),position:bottom,scale:(type:linear),show:!t,style:(),title:(),type:category)),grid:(categoryLines:!f),legendPosition:right,seriesParams:!((data:(id:'1',label:Count),drawLinesBetweenPoints:!t,mode:stacked,show:true,showCircles:!t,type:histogram,valueAxis:ValueAxis-1)),times:!(),type:histogram,valueAxes:!((id:ValueAxis-1,labels:(filter:!f,rotate:0,show:!t,truncate:100),name:LeftAxis-1,position:left,scale:(mode:percentage,type:linear),show:!t,style:(),title:(text:Count),type:value))),title:'New%20Visualization',type:histogram))


\href{https://home.cern/}{home.cern} is the most public website at CERN.
Out of 1043 Drupal websites currently hosted, it alone serves 32\% of monthly unique visitors.
The top 10 websites together serve 79\% of all unique visitors, leaving only 1/5 among them headed for the other 1033 websites.
This is an intrinsic characteristic of the service load, which is heavily skewed towards a very small number of critical websites.

Unique visitors is a good measure of a website's publicity, but a measure that is easier to assess an infrastructure on is the number of HTTP requests.
The situation there is similar, with the top 10 websites being the target of 60\% of all requests.
In section \ref{sec-experiment} we will describe an experiment on resource optimization by assigning websites to different Quality of Service classes,
according to their needs, where both metrics are relevant.


\subsubsection{Concurrent requests}

< graph KIBANA simultaneous requests across infra> %  https://monit-timber-drupal-prod8.cern.ch/kibana/app/kibana#/visualize/create?type=histogram&indexPattern=d2ef41c0-e515-11e9-886e-69c303354aa9&_g=(refreshInterval:(pause:!t,value:0),time:(from:now-2500s,to:now-2250s))&_a=(filters:!(),linked:!f,query:(language:kuery,query:'data.program:%22httpd%22'),uiState:(),vis:(aggs:!((enabled:!t,id:'1',params:(),schema:metric,type:count),(enabled:!t,id:'2',params:(customInterval:'2h',drop_partials:!f,extended_bounds:(),field:metadata.timestamp,interval:s,min_doc_count:1,timeRange:(from:now-1h,to:now),useNormalizedEsInterval:!t),schema:segment,type:date_histogram)),params:(addLegend:!t,addTimeMarker:!f,addTooltip:!t,categoryAxes:!((id:CategoryAxis-1,labels:(show:!t,truncate:100),position:bottom,scale:(type:linear),show:!t,style:(),title:(),type:category)),grid:(categoryLines:!f),legendPosition:right,seriesParams:!((data:(id:'1',label:Count),drawLinesBetweenPoints:!t,mode:stacked,show:true,showCircles:!t,type:histogram,valueAxis:ValueAxis-1)),times:!(),type:histogram,valueAxes:!((id:ValueAxis-1,labels:(filter:!f,rotate:0,show:!t,truncate:100),name:LeftAxis-1,position:left,scale:(mode:normal,type:linear),show:!t,style:(),title:(text:Count),type:value))),title:'New%20Visualization',type:histogram))

The entire infrastructure in regular situations handles peaks of up to 250 concurrent requests per second.


% NOTES
Tell a story about who visits the Drupal infra.
Show Kibana graphs:
- as a whole
- for top 10 websites
- for home.cern
- per minute (simultaneous)
=> define num of users for QoS classes
% NOTES

\subsection{Users of the infrastructure}

Although the load to the infrastructure comprises website visitors, the users that define the functional requirements of this system are website administrators: site builders and web developers.

Technical and non-technical users.
Pick smt from White Area talk.

\subsection{Existing infrastructure}
\label{old-infra}

- physical linux servers with NAS
- haproxy -> nginx -> php-fpm
- static physical size
- poor site isolation: multisite drupal with linux users

\section{Current implementation}
\label{sec-phys-infra}
The infrastructure that currently serves the Drupal websites comprises 8 large physical Linux servers and a NAS shared filesystem.
It runs on CENTOS 7 and uses Puppet as configuration management system.
All servers run the same environment with Systemd services.
Major services are:
\begin{itemize}
    \item HAProxy load balancer: routes requests to worker nodes, with an affinity cookie
    \item Keepalived: implementation of floating IP for the load balancer
    \item Apache httpd: serves Drupal PHP code, WebDAV interface and a few additional PHP management applications
    \item php-fpm: maintains a pool of worker processes that generate Drupal content
\end{itemize}

\subsubsection*{Request journey}

When a request is made to a website on the infrastructure, eg home.cern, DNS resolves the name to the Drupal load balancer floating IP.
HAProxy then (hosted on 1 of the 8 servers) routes the request to one of the nodes.
Apache serves the request with Drupal PHP code. Drupal is configured to look up a directory for each site (multisite).
This flow can be seen in figure \ref{fig:drupal-physical-request-journey} together with the architecture.

Production websites respond to the load by spawning PHP workers, up to a maximum of 25.
A worker process is always listening for requests, even without load.
Test websites, on the other hand, spawn the first worker on demand and scale up to a maximum of 10 workers.
The PHP memory limit for every website is 512MB.

Websites have 2 data components: a directory and a database.
The directory lives on the NAS, shared among all servers.
The database is provided by an external dedicated service, Database on Demand (DBOD).

\begin{figure}[ht]
    \centering
    \hspace{-2em}
    \includegraphics[width=\textwidth]{figures/drupal-physical-request-journey}
    \caption{\emph{Request journey in the physical infrastructure}.
    The current physical infrastructure consists of 8 physical Linux servers with a shared NAS filesystem.
    The {\color{blue} datapath} to access a site is shown in blue.
    The website's {\color{green} 2 data components}: a persistent directory on the NAS and a database on an external service (DBOD).
    }
    \label{fig:drupal-physical-request-journey}
\end{figure}

\subsubsection*{Website isolation}

Each website is assigned a Linux user.
Its directory is owned by it and not accessible by the users of other websites.
When Apache serves a request, it chroots the PHP process into the Drupal directory and sets the website's user.
This distinction provides a basic isolation mechanism.

Nevertheless, \emph{site isolation is relatively weak}.
We've never detected a cross-site security incident,
but there are no cgroup limits to resources, and not enough security layers to defend against privilege escalation exploits.
This is critical concern, given the vulnerability of CMS software \cite{shteiman_why_2014},
and the impact that defacing a high-traffic public site would have to CERN's reputation.

Furthermore, the website environment is inflexible.
All websites locked to the same Drupal version means massive, forced upgrade campaigns with little in the way of testing.
Better integration of testing and development environments is also lacking.

To top off the argument, the Drupal community is deprecating the multi-site functionality, % TODO ref
forcing us to adapt.

\subsection{Limitations of the current infrastructure}

Reiterating the discussion, these are the major limitations of the current infrastructure.
The Kubernetes infrastructure lifts all of them.

\begin{itemize}
    \item Hard to adjust resources, resulting in massive under-utilization
    \item Weak site isolation increases the risk of severe security incidents affecting multiple sites
    \item Inflexible website environment limits development \& testing workflows, makes upgrades cumbersome 
    \item \emph{Technical debt}: a lot of homebrew components built with legacy technologies specific to this system
\end{itemize}


\section{Pilot Kubernetes implementation}
\label{sec-k8s-design}
The system that is described in this section is currently in pilot phase and is planned to \emph{replace} the infrastructure of section \ref{sec-phys-infra} in Q2 2021.

The physical servers are replaced with a cluster of virtual machines composing an Openshift Kubernetes (OKD 4.6) cluster.
Openshift is a Kubernetes distribution that extends vanilla Kubernetes with additional APIs and comes bundled with preconfigured components for infrastructure functionality:
UI friendly to the end users, standardized multitenancy and ingress controllers, Prometheus
% TODO rajula
% Add sophisticated description of Openshift

Instead of the Apache vhost and a part of the PHP-FPM pool, each website is served by 1 or multiple replicas of a pod,
with an Nginx and PHP-FPM containers.
The journey of an HTTP request through the new infrastructure, shown in figure \ref{fig:drupal-physical-request-journey},
holds a strong analogy to the journey through the physical infrastructure in figure \ref{fig:drupal-physical-request-journey}.

\begin{figure}[ht]
    \centering
    \includegraphics[width=\textwidth]{figures/drupal-k8s-request-journey}
    \caption{Caption}
    \label{fig:drupal-k8s-request-journey}
\end{figure}

\subsubsection*{Website management functions}

The infrastructure can be seen as an application that offers its users functions to manage websites.
It provides an API for website admins to specify the kind of website they need: what version of Drupal, what amount of resources, which git repository to fetch configuration from.
Website admins should similarly be able to create new environments of their website for development or test purposes,
clone data between websites, and take and restore backups.

CMS-based websites are inherently stateful applications: end users interact with them constantly and progress their state,
which is stored in a database and in a persistent volume.
To facilitate development, the state of a Drupal application can be "cloned" into another Drupal website.
The development workflow includes ... % TODO

\subsection{Operator pattern}

The design of the management application hinges on Kubernetes controllers and APIs.


\subsection{System components}

\begin{figure}[ht]
    \centering
    \includegraphics[width=\textwidth]{figures/drupal-k8s-architecture}
    \caption{Caption}
    \label{fig:my_label}
\end{figure}


%%% NOTES %%%

diagrams
- [x] serve request journey
- [v] architecture
- admin workflow to update Drupal version
- build configs / auto triggers (maybe merge above)

% TODO
Analyse the "okd4-infrastructure" components.
Describe the authz-operator, custom ingress, cephfs, velero.

%/subsection{Operator pattern}

https://www.openshift.com/learn/topics/operators

Kubernetes design principles, operators


L7 cloud load balancing for .cern domains?

\section{Measuring baseline resource requirements}
\label{sec-experiment}
\newenvironment{conditions}
  {\par\vspace{\abovedisplayskip}\noindent\begin{tabular}{>{$}l<{$} @{${}={}$} l}}
  {\end{tabular}\par\vspace{\belowdisplayskip}}

In order to deploy the new infrastructure, we first need an estimation of resources that Kubernetes will need in order to handle the same load as the physical infrastructure.
To make this estimation, we have stress tested each Quality of Service (QoS) and monitored the resources consumption. 

\subsection{Service level indicators}

In order to be able to answer each Quality of Service (QoS) differently, we have specified some metrics to help resource optimization without ignoring performance.
The defined metric that will be discussed here and will vary among each QoS is the capability of answering requests per second.

Using the physical infra as starting point (fig. \ref{fig:website_bandwidth}), we can observe the requests per second peaked on the most popular website, with around 16000 requests in an hour which averages on 4.4 requests per second.
We, therefore, have defined that there are three type of websites that translates into three types of Quality of Service:
\begin{itemize}
    \item \textbf{Critical} websites, these are the most popular and therefore the most important to have high capability of requests, the defined value is to handle 35 requests per second (Around 8 times the average on peak usage).
    \item \textbf{Standard} websites, these usually don't face as much traffic and therefore don't need to have high capability of requests, the defined value is to handle 10 requests per second. 
    \item \textbf{Test} websites, as in the name itself, these are used to test new features or add new content by website managers, and therefore are used by testers and developers, these type of websites only need to handle few requests, therefore the defined value is to handle 2 requests per second.
\end{itemize}

%- entire infra: 250 req/sec
%- critical website: 35 req/sec
%- standard website: 5-10 req/sec % TODO set final value
%- test website: 2 max threads

%- Find configuration that is able to handle specified req/sec load
%- Two Graphs (1 for critical and 1 normal), showing AVG response time plots + Request per second plots
%- See the Resources usage for each configuration
%- Make an estimation of total resources , note that critical websites will have $(MAX_LOAD) + 2(IDLE_LOAD)$ since they will have 3 pods

%Total number of Critical websites: 20
%Total number of Standard websites: 600
%Total number of Test websites: 500
%- Make an estimation of total infra resources ( sum to the estimation of total resources with openshift-* elements)

% #### Resources used in the physical infra:
% - Memory: 2TB
% - CPU: 512

\subsection{Stress test setup}

% (we have provioned websites and migrated data from the old infra)
To have a proper setup to test, we have migrated some websites with diversified content to the new infrastructure to observe how websites already hosted under the Physical infra perform on the Kubernetes infrastructure.

%To evaluate the behavior of our Kubernetes infrastructure, with the goal of analysing the new infrastructure's capability of answering the same needs with fewer resources, the following experiment was conducted.

%The following elements were required, 

The experiment has a dedicated Kubernetes cluster that deploys a custom tool based on \hyperlink{locust.io}{Locust} to make multiple requests to the targeted website on the new infrastructure. 
The simulation of requests is done by running multiple \hyperlink{https://kubernetes.io/docs/concepts/workloads/pods/}{Pods}, each containing multiple processes that will request URLs at random\footnote{It iterates urls through the website to discover all the URLs, and only after having all the URLs, it connects to one of them at random}.

\subsubsection{Stress load}
Multiple runs have been made with different configurations in order to find a suitable one for each QoS to process the desired requests per second with minimal resource consumption.

We observed that after a number of parallel users in the same pod, the users would get bottle-necked by Pod network constraints. So for our tests we observer that each Pod could have no more than 5 processes before network constraints would interfere.

%Important notes:
%- RAM consumption before first connection is ~11MiB, 500 to 600MiB after (even if no accesses are being done)
%- Time used for a stress run was decided to be 6min because it's enough to see the stabilized response time

\subsubsection{Measurements}

The following graph shows the average response time from the client's side as well the requests per second handled from the server side at the same time. 

The stress tests do a ramp up during the first minute, after which they maintain the stress load for more 9 minutes.
Thus, the total duration for each run is 10minutes, this value was considered enough to present the trend that would follow in a bigger time frame.

\includegraphics[width=400]{figures/experiment-figures/critical_run.png}

During this run, the resources consumption was also monitored, here we can see the usage for `nursery` and `fluka` websites respectively:

\includegraphics[width=\linewidth/2]{figures/experiment-figures/nursery_memory_usage.jpg}
\includegraphics[width=\linewidth/2]{figures/experiment-figures/fluka_memory_usage.jpg}

The same experiment was conducted to the same websites with configurations based on the other QoS. 
The following tables show the highest response time under full stress and lowest requests per second for each QoS.

\begin{table}[!htb]
\centering

\begin{tabular}{|l|ll|ll}
\cline{1-3}
\textbf{website} & \multicolumn{1}{l|}{\textbf{Stress (req/s)}} & \textbf{Response (ms)} &  &  \\ \cline{1-3}
nurseryschool    & 10                                           & 510                    &  &  \\ \cline{1-1}
fluka            & 26                                           & 570                    &  &  \\ \cline{1-3}
\end{tabular}
\caption{Test QoS Values}


\begin{tabular}{|l|ll|ll}
\cline{1-3}
\textbf{website} & \multicolumn{1}{l|}{\textbf{Stress (req/s)}} & \textbf{Response (ms)} &  &  \\ \cline{1-3}
nurseryschool    & 18                                           & 240                    &  &  \\ \cline{1-1}
fluka            & 48                                           & 103                    &  &  \\ \cline{1-3}
\end{tabular}
\hfill
\caption{Standard QoS values}
\begin{tabular}{|l|ll|ll}
\cline{1-3}
\textbf{website} & \multicolumn{1}{l|}{\textbf{Stress (req/s)}} & \textbf{Response (ms)} &  &  \\ \cline{1-3}
nurseryschool    & 41  & 870   &  &  \\ \cline{1-1}
fluka            & 74  & 600   &  &  \\ \cline{1-3}
\end{tabular}
\caption{Critical QoS values}
\end{table}


%Table \ref{tabel_of_resources}
\begin{table}[]
\centering
\label{tabel_of_resources}
\begin{tabular}{|l|ll|ll}
\cline{1-3}
\textbf{QoS} & \multicolumn{1}{l|}{\textbf{CPU}} & \textbf{RAM(MiB)} &  &  \\ \cline{1-3}
test         & 0.3                                & 104               &  &  \\ \cline{1-1}
standard     & 2.3                                & 257               &  &  \\ \cline{1-1}
critical     & 3  & 800               &  &  \\ \cline{1-3}
\end{tabular}
\caption{ Resources peak consumption per QoS.}
\end{table}
% Show that we have STABLE response times

\subsection{Results: resource provision}

Based on the results seen in measurements section, we can now make an estimation of resource provision for the new infrastructure based on the values retrieved.

For expected memory:
\[
\begin{array}{rcl}
TotalMem & = & C * L + 2* C * I + S * L + T * L \\
 & = & 20 * 960MiB + 20*104MiB + 600*309MiB + 500*125MiB \\
 & = & 269180MiB  \approx  282.3GB
\end{array}
\]
Where:
\begin{conditions}
Expected Load  &  Max Seen load + 20\% overhead \\
 L     &  Expected Max Load for specific QoS website \\
 I     &  Expected Idle Load for specific QoS website \\   
 C     &  Total number of Critical Websites \\
 S     & Total number of Standard Websites \\
 T     & Total number of Test Websites \\
\end{conditions}

On the physical infrastructure we have provisioned 2TB of memory, on the new Kubernetes infrastructure we have a rough estimation on total memory used if all websites are under load is of about 283GB. The prediction expects to only have 14.15\% of the memory required in the physical infrastructure.


For expected cpu:
\[
\begin{array}{rcl}
TotalCPU & = & C * L + 2* C * I + S * L + T * L \\
 & = & 20 * 3VCpu + 2*0.01VCpu + 600* 1.6VCpu + 500* 1VCpu \\
 & = & FINALRESULT
\end{array}
\]




\section{Discussion}
\label{sec-discussion}
\subsubsection*{Summary: what were our main points}

- reuse across web frameworks
- extensibility to different Content Management Systems

\subsubsection*{Limitations of this work, what did we not put in because of space / pilot phase}

- design not complete (DBOD integration, application metrics and log processing)
- migrations in Q2 2021

\subsubsection*{Directions to explore}

- support additional CMS (WordPress) as a Service
- deeper observability inside PHP
- metrics/log processing and root cause analysis
- intrusion detection

\section*{Acknowledgements}

This work relies upon the contributions of all members of the Web Frameworks section in the CDA group of CERN's IT department,
especially those that designed and implemented the common components.
We thank specifically Alex Lossent, Ismael Posada Trobo, Joao Esteves Marcal, Iago Santos Pardo, Aleksandra Wardzinska, Michal Kolodziejski and Emmanouil Fokas.

\bibliography{references.bib}
\end{document}