It is striking how big a difference it makes to discuss a design with 10 engineers rather than 3, and to have the peace of mind in case of emergency that many colleagues can take part.
This is the hidden benefit of sharing a common platform, which this team has already felt.
Especially in CERN's dynamic environment where the turnover of people is high, knowledge silos can be ill afforded.

The Pilot phase of this new infrastructure is still too immature to provide significant operational insights.
We fully expect a few minor adaptations of the presented design as we scale up to production, but the experiments of
section \ref{sec-experiment} reassure us that the infrastructure is functional and that modifications should be limited to the details.
On the other hand, all the limitations of the physical infrastructure of section \ref{sec-limitations} have been lifted.

The next big challenge will be the production migrations in Q2 2021.
Development is ongoing, including critical features for production.
It is important to keep them as transparent to the website admins as possible and minimize disruption.
From the power users however we expect the new features to receive a warm welcome and open new doors in their workflows.

\subsubsection*{Directions to explore}

Develop once, run everywhere is a yet-to-be materialized promise.
Plans for disaster recovery from a catastrophic failure of the CERN data center hinge on maintaining a public communications channel accessible.
With Kubernetes cluster federation, using Public Cloud resources as a safety net is conceivable.

We will explore adding WordPress as a Service to the same infrastructure.
Fundamentally, it should take no more than introducing a new build configuration.

The \href{https://landscape.cncf.io/}{CNCF landscape} provides a salient overview of industry-standard solutions that can take this project the proverbial extra "mile".
We plan to experiment with runtime security, root cause analysis, chaos engineering and serverless for the non-production environments.
Kubernetes turns a homebrew system into a cosmopolitan denizen of a brave new world of the Cloud.