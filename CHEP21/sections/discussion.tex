It makes a big difference to discuss a design with 10 engineers rather than 3, and to have the peace of mind in case of emergency that many colleagues can take part.
This is the hidden benefit of sharing a common platform.
Especially in CERN's dynamic environment where the turnover of people is high, knowledge silos can be ill afforded.

The Pilot phase of this new infrastructure is still too immature to provide significant operational insights.
We iterated on many design choices: the way to mark releases of the CERN Drupal Distribution, the \texttt{DrupalSite} update logic, the Cephfs integration (remote storage)...
Now we consider the design stable, while focusing development on Drupal-specific aspects, such as SSO integration.
Functionality has been verified by copying and instantiating in Kubernetes many websites from the physical infrastructure,
while the experiments of section \ref{sec-experiment} validate basic performance requirements.
On the other hand, all the limitations of the physical infrastructure of section \ref{sec-limitations} have been lifted.

The next big challenge will be the production migrations in summer 2021.
It is important to keep them as transparent to the website admins as possible and minimize disruption.
From the power users however we expect the new features to receive a warm welcome and open new doors in their workflows.

\subsubsection*{Directions to explore}

Develop once, run everywhere is a yet-to-be materialized promise.
Plans for disaster recovery from a catastrophic failure of the CERN data center hinge on maintaining a public communications channel accessible.
With Kubernetes cluster federation, using Public Cloud resources as a safety net is conceivable.

We will explore adding WordPress as a Service to the same infrastructure.
Fundamentally, it should take no more than introducing a new build configuration.

The CNCF landscape \cite{cncfLandscape} provides a salient overview of industry-standard solutions that can take this project the proverbial extra "mile".
We plan to experiment with runtime security, root cause analysis, chaos engineering and serverless for the non-production environments.
Kubernetes turns a homebrew system into a cosmopolitan denizen of the brave new world of the Cloud.